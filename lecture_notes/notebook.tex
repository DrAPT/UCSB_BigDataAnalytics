
% Default to the notebook output style

    


% Inherit from the specified cell style.




    
\documentclass[11pt]{article}

    
    
    \usepackage[T1]{fontenc}
    % Nicer default font (+ math font) than Computer Modern for most use cases
    \usepackage{mathpazo}

    % Basic figure setup, for now with no caption control since it's done
    % automatically by Pandoc (which extracts ![](path) syntax from Markdown).
    \usepackage{graphicx}
    % We will generate all images so they have a width \maxwidth. This means
    % that they will get their normal width if they fit onto the page, but
    % are scaled down if they would overflow the margins.
    \makeatletter
    \def\maxwidth{\ifdim\Gin@nat@width>\linewidth\linewidth
    \else\Gin@nat@width\fi}
    \makeatother
    \let\Oldincludegraphics\includegraphics
    % Set max figure width to be 80% of text width, for now hardcoded.
    \renewcommand{\includegraphics}[1]{\Oldincludegraphics[width=.8\maxwidth]{#1}}
    % Ensure that by default, figures have no caption (until we provide a
    % proper Figure object with a Caption API and a way to capture that
    % in the conversion process - todo).
    \usepackage{caption}
    \DeclareCaptionLabelFormat{nolabel}{}
    \captionsetup{labelformat=nolabel}

    \usepackage{adjustbox} % Used to constrain images to a maximum size 
    \usepackage{xcolor} % Allow colors to be defined
    \usepackage{enumerate} % Needed for markdown enumerations to work
    \usepackage{geometry} % Used to adjust the document margins
    \usepackage{amsmath} % Equations
    \usepackage{amssymb} % Equations
    \usepackage{textcomp} % defines textquotesingle
    % Hack from http://tex.stackexchange.com/a/47451/13684:
    \AtBeginDocument{%
        \def\PYZsq{\textquotesingle}% Upright quotes in Pygmentized code
    }
    \usepackage{upquote} % Upright quotes for verbatim code
    \usepackage{eurosym} % defines \euro
    \usepackage[mathletters]{ucs} % Extended unicode (utf-8) support
    \usepackage[utf8x]{inputenc} % Allow utf-8 characters in the tex document
    \usepackage{fancyvrb} % verbatim replacement that allows latex
    \usepackage{grffile} % extends the file name processing of package graphics 
                         % to support a larger range 
    % The hyperref package gives us a pdf with properly built
    % internal navigation ('pdf bookmarks' for the table of contents,
    % internal cross-reference links, web links for URLs, etc.)
    \usepackage{hyperref}
    \usepackage{longtable} % longtable support required by pandoc >1.10
    \usepackage{booktabs}  % table support for pandoc > 1.12.2
    \usepackage[inline]{enumitem} % IRkernel/repr support (it uses the enumerate* environment)
    \usepackage[normalem]{ulem} % ulem is needed to support strikethroughs (\sout)
                                % normalem makes italics be italics, not underlines
    

    
    
    % Colors for the hyperref package
    \definecolor{urlcolor}{rgb}{0,.145,.698}
    \definecolor{linkcolor}{rgb}{.71,0.21,0.01}
    \definecolor{citecolor}{rgb}{.12,.54,.11}

    % ANSI colors
    \definecolor{ansi-black}{HTML}{3E424D}
    \definecolor{ansi-black-intense}{HTML}{282C36}
    \definecolor{ansi-red}{HTML}{E75C58}
    \definecolor{ansi-red-intense}{HTML}{B22B31}
    \definecolor{ansi-green}{HTML}{00A250}
    \definecolor{ansi-green-intense}{HTML}{007427}
    \definecolor{ansi-yellow}{HTML}{DDB62B}
    \definecolor{ansi-yellow-intense}{HTML}{B27D12}
    \definecolor{ansi-blue}{HTML}{208FFB}
    \definecolor{ansi-blue-intense}{HTML}{0065CA}
    \definecolor{ansi-magenta}{HTML}{D160C4}
    \definecolor{ansi-magenta-intense}{HTML}{A03196}
    \definecolor{ansi-cyan}{HTML}{60C6C8}
    \definecolor{ansi-cyan-intense}{HTML}{258F8F}
    \definecolor{ansi-white}{HTML}{C5C1B4}
    \definecolor{ansi-white-intense}{HTML}{A1A6B2}

    % commands and environments needed by pandoc snippets
    % extracted from the output of `pandoc -s`
    \providecommand{\tightlist}{%
      \setlength{\itemsep}{0pt}\setlength{\parskip}{0pt}}
    \DefineVerbatimEnvironment{Highlighting}{Verbatim}{commandchars=\\\{\}}
    % Add ',fontsize=\small' for more characters per line
    \newenvironment{Shaded}{}{}
    \newcommand{\KeywordTok}[1]{\textcolor[rgb]{0.00,0.44,0.13}{\textbf{{#1}}}}
    \newcommand{\DataTypeTok}[1]{\textcolor[rgb]{0.56,0.13,0.00}{{#1}}}
    \newcommand{\DecValTok}[1]{\textcolor[rgb]{0.25,0.63,0.44}{{#1}}}
    \newcommand{\BaseNTok}[1]{\textcolor[rgb]{0.25,0.63,0.44}{{#1}}}
    \newcommand{\FloatTok}[1]{\textcolor[rgb]{0.25,0.63,0.44}{{#1}}}
    \newcommand{\CharTok}[1]{\textcolor[rgb]{0.25,0.44,0.63}{{#1}}}
    \newcommand{\StringTok}[1]{\textcolor[rgb]{0.25,0.44,0.63}{{#1}}}
    \newcommand{\CommentTok}[1]{\textcolor[rgb]{0.38,0.63,0.69}{\textit{{#1}}}}
    \newcommand{\OtherTok}[1]{\textcolor[rgb]{0.00,0.44,0.13}{{#1}}}
    \newcommand{\AlertTok}[1]{\textcolor[rgb]{1.00,0.00,0.00}{\textbf{{#1}}}}
    \newcommand{\FunctionTok}[1]{\textcolor[rgb]{0.02,0.16,0.49}{{#1}}}
    \newcommand{\RegionMarkerTok}[1]{{#1}}
    \newcommand{\ErrorTok}[1]{\textcolor[rgb]{1.00,0.00,0.00}{\textbf{{#1}}}}
    \newcommand{\NormalTok}[1]{{#1}}
    
    % Additional commands for more recent versions of Pandoc
    \newcommand{\ConstantTok}[1]{\textcolor[rgb]{0.53,0.00,0.00}{{#1}}}
    \newcommand{\SpecialCharTok}[1]{\textcolor[rgb]{0.25,0.44,0.63}{{#1}}}
    \newcommand{\VerbatimStringTok}[1]{\textcolor[rgb]{0.25,0.44,0.63}{{#1}}}
    \newcommand{\SpecialStringTok}[1]{\textcolor[rgb]{0.73,0.40,0.53}{{#1}}}
    \newcommand{\ImportTok}[1]{{#1}}
    \newcommand{\DocumentationTok}[1]{\textcolor[rgb]{0.73,0.13,0.13}{\textit{{#1}}}}
    \newcommand{\AnnotationTok}[1]{\textcolor[rgb]{0.38,0.63,0.69}{\textbf{\textit{{#1}}}}}
    \newcommand{\CommentVarTok}[1]{\textcolor[rgb]{0.38,0.63,0.69}{\textbf{\textit{{#1}}}}}
    \newcommand{\VariableTok}[1]{\textcolor[rgb]{0.10,0.09,0.49}{{#1}}}
    \newcommand{\ControlFlowTok}[1]{\textcolor[rgb]{0.00,0.44,0.13}{\textbf{{#1}}}}
    \newcommand{\OperatorTok}[1]{\textcolor[rgb]{0.40,0.40,0.40}{{#1}}}
    \newcommand{\BuiltInTok}[1]{{#1}}
    \newcommand{\ExtensionTok}[1]{{#1}}
    \newcommand{\PreprocessorTok}[1]{\textcolor[rgb]{0.74,0.48,0.00}{{#1}}}
    \newcommand{\AttributeTok}[1]{\textcolor[rgb]{0.49,0.56,0.16}{{#1}}}
    \newcommand{\InformationTok}[1]{\textcolor[rgb]{0.38,0.63,0.69}{\textbf{\textit{{#1}}}}}
    \newcommand{\WarningTok}[1]{\textcolor[rgb]{0.38,0.63,0.69}{\textbf{\textit{{#1}}}}}
    
    
    % Define a nice break command that doesn't care if a line doesn't already
    % exist.
    \def\br{\hspace*{\fill} \\* }
    % Math Jax compatability definitions
    \def\gt{>}
    \def\lt{<}
    % Document parameters
    \title{07. Machine Learning with MLlib - Overview}
    
    
    

    % Pygments definitions
    
\makeatletter
\def\PY@reset{\let\PY@it=\relax \let\PY@bf=\relax%
    \let\PY@ul=\relax \let\PY@tc=\relax%
    \let\PY@bc=\relax \let\PY@ff=\relax}
\def\PY@tok#1{\csname PY@tok@#1\endcsname}
\def\PY@toks#1+{\ifx\relax#1\empty\else%
    \PY@tok{#1}\expandafter\PY@toks\fi}
\def\PY@do#1{\PY@bc{\PY@tc{\PY@ul{%
    \PY@it{\PY@bf{\PY@ff{#1}}}}}}}
\def\PY#1#2{\PY@reset\PY@toks#1+\relax+\PY@do{#2}}

\expandafter\def\csname PY@tok@w\endcsname{\def\PY@tc##1{\textcolor[rgb]{0.73,0.73,0.73}{##1}}}
\expandafter\def\csname PY@tok@c\endcsname{\let\PY@it=\textit\def\PY@tc##1{\textcolor[rgb]{0.25,0.50,0.50}{##1}}}
\expandafter\def\csname PY@tok@cp\endcsname{\def\PY@tc##1{\textcolor[rgb]{0.74,0.48,0.00}{##1}}}
\expandafter\def\csname PY@tok@k\endcsname{\let\PY@bf=\textbf\def\PY@tc##1{\textcolor[rgb]{0.00,0.50,0.00}{##1}}}
\expandafter\def\csname PY@tok@kp\endcsname{\def\PY@tc##1{\textcolor[rgb]{0.00,0.50,0.00}{##1}}}
\expandafter\def\csname PY@tok@kt\endcsname{\def\PY@tc##1{\textcolor[rgb]{0.69,0.00,0.25}{##1}}}
\expandafter\def\csname PY@tok@o\endcsname{\def\PY@tc##1{\textcolor[rgb]{0.40,0.40,0.40}{##1}}}
\expandafter\def\csname PY@tok@ow\endcsname{\let\PY@bf=\textbf\def\PY@tc##1{\textcolor[rgb]{0.67,0.13,1.00}{##1}}}
\expandafter\def\csname PY@tok@nb\endcsname{\def\PY@tc##1{\textcolor[rgb]{0.00,0.50,0.00}{##1}}}
\expandafter\def\csname PY@tok@nf\endcsname{\def\PY@tc##1{\textcolor[rgb]{0.00,0.00,1.00}{##1}}}
\expandafter\def\csname PY@tok@nc\endcsname{\let\PY@bf=\textbf\def\PY@tc##1{\textcolor[rgb]{0.00,0.00,1.00}{##1}}}
\expandafter\def\csname PY@tok@nn\endcsname{\let\PY@bf=\textbf\def\PY@tc##1{\textcolor[rgb]{0.00,0.00,1.00}{##1}}}
\expandafter\def\csname PY@tok@ne\endcsname{\let\PY@bf=\textbf\def\PY@tc##1{\textcolor[rgb]{0.82,0.25,0.23}{##1}}}
\expandafter\def\csname PY@tok@nv\endcsname{\def\PY@tc##1{\textcolor[rgb]{0.10,0.09,0.49}{##1}}}
\expandafter\def\csname PY@tok@no\endcsname{\def\PY@tc##1{\textcolor[rgb]{0.53,0.00,0.00}{##1}}}
\expandafter\def\csname PY@tok@nl\endcsname{\def\PY@tc##1{\textcolor[rgb]{0.63,0.63,0.00}{##1}}}
\expandafter\def\csname PY@tok@ni\endcsname{\let\PY@bf=\textbf\def\PY@tc##1{\textcolor[rgb]{0.60,0.60,0.60}{##1}}}
\expandafter\def\csname PY@tok@na\endcsname{\def\PY@tc##1{\textcolor[rgb]{0.49,0.56,0.16}{##1}}}
\expandafter\def\csname PY@tok@nt\endcsname{\let\PY@bf=\textbf\def\PY@tc##1{\textcolor[rgb]{0.00,0.50,0.00}{##1}}}
\expandafter\def\csname PY@tok@nd\endcsname{\def\PY@tc##1{\textcolor[rgb]{0.67,0.13,1.00}{##1}}}
\expandafter\def\csname PY@tok@s\endcsname{\def\PY@tc##1{\textcolor[rgb]{0.73,0.13,0.13}{##1}}}
\expandafter\def\csname PY@tok@sd\endcsname{\let\PY@it=\textit\def\PY@tc##1{\textcolor[rgb]{0.73,0.13,0.13}{##1}}}
\expandafter\def\csname PY@tok@si\endcsname{\let\PY@bf=\textbf\def\PY@tc##1{\textcolor[rgb]{0.73,0.40,0.53}{##1}}}
\expandafter\def\csname PY@tok@se\endcsname{\let\PY@bf=\textbf\def\PY@tc##1{\textcolor[rgb]{0.73,0.40,0.13}{##1}}}
\expandafter\def\csname PY@tok@sr\endcsname{\def\PY@tc##1{\textcolor[rgb]{0.73,0.40,0.53}{##1}}}
\expandafter\def\csname PY@tok@ss\endcsname{\def\PY@tc##1{\textcolor[rgb]{0.10,0.09,0.49}{##1}}}
\expandafter\def\csname PY@tok@sx\endcsname{\def\PY@tc##1{\textcolor[rgb]{0.00,0.50,0.00}{##1}}}
\expandafter\def\csname PY@tok@m\endcsname{\def\PY@tc##1{\textcolor[rgb]{0.40,0.40,0.40}{##1}}}
\expandafter\def\csname PY@tok@gh\endcsname{\let\PY@bf=\textbf\def\PY@tc##1{\textcolor[rgb]{0.00,0.00,0.50}{##1}}}
\expandafter\def\csname PY@tok@gu\endcsname{\let\PY@bf=\textbf\def\PY@tc##1{\textcolor[rgb]{0.50,0.00,0.50}{##1}}}
\expandafter\def\csname PY@tok@gd\endcsname{\def\PY@tc##1{\textcolor[rgb]{0.63,0.00,0.00}{##1}}}
\expandafter\def\csname PY@tok@gi\endcsname{\def\PY@tc##1{\textcolor[rgb]{0.00,0.63,0.00}{##1}}}
\expandafter\def\csname PY@tok@gr\endcsname{\def\PY@tc##1{\textcolor[rgb]{1.00,0.00,0.00}{##1}}}
\expandafter\def\csname PY@tok@ge\endcsname{\let\PY@it=\textit}
\expandafter\def\csname PY@tok@gs\endcsname{\let\PY@bf=\textbf}
\expandafter\def\csname PY@tok@gp\endcsname{\let\PY@bf=\textbf\def\PY@tc##1{\textcolor[rgb]{0.00,0.00,0.50}{##1}}}
\expandafter\def\csname PY@tok@go\endcsname{\def\PY@tc##1{\textcolor[rgb]{0.53,0.53,0.53}{##1}}}
\expandafter\def\csname PY@tok@gt\endcsname{\def\PY@tc##1{\textcolor[rgb]{0.00,0.27,0.87}{##1}}}
\expandafter\def\csname PY@tok@err\endcsname{\def\PY@bc##1{\setlength{\fboxsep}{0pt}\fcolorbox[rgb]{1.00,0.00,0.00}{1,1,1}{\strut ##1}}}
\expandafter\def\csname PY@tok@kc\endcsname{\let\PY@bf=\textbf\def\PY@tc##1{\textcolor[rgb]{0.00,0.50,0.00}{##1}}}
\expandafter\def\csname PY@tok@kd\endcsname{\let\PY@bf=\textbf\def\PY@tc##1{\textcolor[rgb]{0.00,0.50,0.00}{##1}}}
\expandafter\def\csname PY@tok@kn\endcsname{\let\PY@bf=\textbf\def\PY@tc##1{\textcolor[rgb]{0.00,0.50,0.00}{##1}}}
\expandafter\def\csname PY@tok@kr\endcsname{\let\PY@bf=\textbf\def\PY@tc##1{\textcolor[rgb]{0.00,0.50,0.00}{##1}}}
\expandafter\def\csname PY@tok@bp\endcsname{\def\PY@tc##1{\textcolor[rgb]{0.00,0.50,0.00}{##1}}}
\expandafter\def\csname PY@tok@fm\endcsname{\def\PY@tc##1{\textcolor[rgb]{0.00,0.00,1.00}{##1}}}
\expandafter\def\csname PY@tok@vc\endcsname{\def\PY@tc##1{\textcolor[rgb]{0.10,0.09,0.49}{##1}}}
\expandafter\def\csname PY@tok@vg\endcsname{\def\PY@tc##1{\textcolor[rgb]{0.10,0.09,0.49}{##1}}}
\expandafter\def\csname PY@tok@vi\endcsname{\def\PY@tc##1{\textcolor[rgb]{0.10,0.09,0.49}{##1}}}
\expandafter\def\csname PY@tok@vm\endcsname{\def\PY@tc##1{\textcolor[rgb]{0.10,0.09,0.49}{##1}}}
\expandafter\def\csname PY@tok@sa\endcsname{\def\PY@tc##1{\textcolor[rgb]{0.73,0.13,0.13}{##1}}}
\expandafter\def\csname PY@tok@sb\endcsname{\def\PY@tc##1{\textcolor[rgb]{0.73,0.13,0.13}{##1}}}
\expandafter\def\csname PY@tok@sc\endcsname{\def\PY@tc##1{\textcolor[rgb]{0.73,0.13,0.13}{##1}}}
\expandafter\def\csname PY@tok@dl\endcsname{\def\PY@tc##1{\textcolor[rgb]{0.73,0.13,0.13}{##1}}}
\expandafter\def\csname PY@tok@s2\endcsname{\def\PY@tc##1{\textcolor[rgb]{0.73,0.13,0.13}{##1}}}
\expandafter\def\csname PY@tok@sh\endcsname{\def\PY@tc##1{\textcolor[rgb]{0.73,0.13,0.13}{##1}}}
\expandafter\def\csname PY@tok@s1\endcsname{\def\PY@tc##1{\textcolor[rgb]{0.73,0.13,0.13}{##1}}}
\expandafter\def\csname PY@tok@mb\endcsname{\def\PY@tc##1{\textcolor[rgb]{0.40,0.40,0.40}{##1}}}
\expandafter\def\csname PY@tok@mf\endcsname{\def\PY@tc##1{\textcolor[rgb]{0.40,0.40,0.40}{##1}}}
\expandafter\def\csname PY@tok@mh\endcsname{\def\PY@tc##1{\textcolor[rgb]{0.40,0.40,0.40}{##1}}}
\expandafter\def\csname PY@tok@mi\endcsname{\def\PY@tc##1{\textcolor[rgb]{0.40,0.40,0.40}{##1}}}
\expandafter\def\csname PY@tok@il\endcsname{\def\PY@tc##1{\textcolor[rgb]{0.40,0.40,0.40}{##1}}}
\expandafter\def\csname PY@tok@mo\endcsname{\def\PY@tc##1{\textcolor[rgb]{0.40,0.40,0.40}{##1}}}
\expandafter\def\csname PY@tok@ch\endcsname{\let\PY@it=\textit\def\PY@tc##1{\textcolor[rgb]{0.25,0.50,0.50}{##1}}}
\expandafter\def\csname PY@tok@cm\endcsname{\let\PY@it=\textit\def\PY@tc##1{\textcolor[rgb]{0.25,0.50,0.50}{##1}}}
\expandafter\def\csname PY@tok@cpf\endcsname{\let\PY@it=\textit\def\PY@tc##1{\textcolor[rgb]{0.25,0.50,0.50}{##1}}}
\expandafter\def\csname PY@tok@c1\endcsname{\let\PY@it=\textit\def\PY@tc##1{\textcolor[rgb]{0.25,0.50,0.50}{##1}}}
\expandafter\def\csname PY@tok@cs\endcsname{\let\PY@it=\textit\def\PY@tc##1{\textcolor[rgb]{0.25,0.50,0.50}{##1}}}

\def\PYZbs{\char`\\}
\def\PYZus{\char`\_}
\def\PYZob{\char`\{}
\def\PYZcb{\char`\}}
\def\PYZca{\char`\^}
\def\PYZam{\char`\&}
\def\PYZlt{\char`\<}
\def\PYZgt{\char`\>}
\def\PYZsh{\char`\#}
\def\PYZpc{\char`\%}
\def\PYZdl{\char`\$}
\def\PYZhy{\char`\-}
\def\PYZsq{\char`\'}
\def\PYZdq{\char`\"}
\def\PYZti{\char`\~}
% for compatibility with earlier versions
\def\PYZat{@}
\def\PYZlb{[}
\def\PYZrb{]}
\makeatother


    % Exact colors from NB
    \definecolor{incolor}{rgb}{0.0, 0.0, 0.5}
    \definecolor{outcolor}{rgb}{0.545, 0.0, 0.0}



    
    % Prevent overflowing lines due to hard-to-break entities
    \sloppy 
    % Setup hyperref package
    \hypersetup{
      breaklinks=true,  % so long urls are correctly broken across lines
      colorlinks=true,
      urlcolor=urlcolor,
      linkcolor=linkcolor,
      citecolor=citecolor,
      }
    % Slightly bigger margins than the latex defaults
    
    \geometry{verbose,tmargin=1in,bmargin=1in,lmargin=1in,rmargin=1in}
    
    

    \begin{document}
    
    
    \maketitle
    
    

    
    \hypertarget{machine-learning-with-mllib}{%
\subsection{Machine Learning with
MLlib}\label{machine-learning-with-mllib}}

\hypertarget{introduction-and-feature-extraction}{%
\subsection{\texorpdfstring{\emph{Introduction and Feature
Extraction}}{Introduction and Feature Extraction}}\label{introduction-and-feature-extraction}}

\hypertarget{university-of-california-santa-barbara}{%
\subsubsection{University of California, Santa
Barbara}\label{university-of-california-santa-barbara}}

\hypertarget{pstat-135235}{%
\subsubsection{PSTAT 135/235}\label{pstat-135235}}

\hypertarget{last-updated-oct-19-2018}{%
\subsubsection{Last Updated: Oct 19,
2018}\label{last-updated-oct-19-2018}}

\begin{center}\rule{0.5\linewidth}{\linethickness}\end{center}

\hypertarget{sources}{%
\subsubsection{Sources}\label{sources}}

\begin{enumerate}
\def\labelenumi{\arabic{enumi}.}
\tightlist
\item
  Learning Spark
\item
  Spark Documentation\\
  https://spark.apache.org/docs/latest/mllib-data-types.html\\
  http://spark.apache.org/docs/1.2.0/mllib-feature-extraction.html
\end{enumerate}

\hypertarget{objectives}{%
\subsubsection{OBJECTIVES}\label{objectives}}

\begin{enumerate}
\def\labelenumi{\arabic{enumi}.}
\tightlist
\item
  Introduction to the machine learning library
\item
  Introduction to MLlib data types
\item
  Discuss Feature Extraction tools in MLLib
\end{enumerate}

\hypertarget{concepts-and-functions}{%
\subsubsection{CONCEPTS AND FUNCTIONS}\label{concepts-and-functions}}

\begin{itemize}
\item
  pipeline\\
\item
  supervised and unsupervised learning\\
\item
  learning tasks: classification, regression, clustering, dimensionality
  reduction\\
\item
  training set, testing set\\
\item
  feature extraction
\item
  MLlib data types:

  \begin{itemize}
  \tightlist
  \item
    LabeledPoint\\
  \item
    sparse vector, dense vector\\
  \item
    sparse matrix, dense matrix\\
  \item
    Rating
  \end{itemize}
\item
  Feature Extraction\\
\item
  TF-IDF\\
\item
  Word2Vec\\
\item
  Cosine Similarity
\end{itemize}

\begin{center}\rule{0.5\linewidth}{\linethickness}\end{center}

    \textbf{MLlib}

Contains Spark's ML library\\
Works on RDDs\\
Contains only algorithms that can be parallelized, since those run well
on clusters\\
Includes a pipeline API useful for building ML pipelines, similar to
scikit-learn

    \hypertarget{build-logreg-classifier-to-predict-spam-vs-not}{%
\subsubsection{Build LogReg Classifier to Predict Spam vs
Not}\label{build-logreg-classifier-to-predict-spam-vs-not}}

    \begin{Verbatim}[commandchars=\\\{\}]
{\color{incolor}In [{\color{incolor}12}]:} \PY{n}{data\PYZus{}path} \PY{o}{=} \PY{l+s+s1}{\PYZsq{}}\PY{l+s+s1}{/home/jovyan/work/data/mllib/}\PY{l+s+s1}{\PYZsq{}}
\end{Verbatim}


    \begin{Verbatim}[commandchars=\\\{\}]
{\color{incolor}In [{\color{incolor}17}]:} \PY{k+kn}{import} \PY{n+nn}{os}
         \PY{k+kn}{from} \PY{n+nn}{pyspark}\PY{n+nn}{.}\PY{n+nn}{mllib}\PY{n+nn}{.}\PY{n+nn}{regression} \PY{k}{import} \PY{n}{LabeledPoint}
         \PY{k+kn}{from} \PY{n+nn}{pyspark}\PY{n+nn}{.}\PY{n+nn}{mllib}\PY{n+nn}{.}\PY{n+nn}{feature} \PY{k}{import} \PY{n}{HashingTF}
         \PY{k+kn}{from} \PY{n+nn}{pyspark}\PY{n+nn}{.}\PY{n+nn}{mllib}\PY{n+nn}{.}\PY{n+nn}{classification} \PY{k}{import} \PY{n}{LogisticRegressionWithSGD}
\end{Verbatim}


    \begin{Verbatim}[commandchars=\\\{\}]
{\color{incolor}In [{\color{incolor}14}]:} \PY{k+kn}{from} \PY{n+nn}{pyspark}\PY{n+nn}{.}\PY{n+nn}{sql} \PY{k}{import} \PY{n}{SparkSession}
         
         \PY{n}{spark} \PY{o}{=} \PY{n}{SparkSession}\PY{o}{.}\PY{n}{builder} \PYZbs{}
                 \PY{o}{.}\PY{n}{master}\PY{p}{(}\PY{l+s+s2}{\PYZdq{}}\PY{l+s+s2}{local}\PY{l+s+s2}{\PYZdq{}}\PY{p}{)} \PYZbs{}
                 \PY{o}{.}\PY{n}{appName}\PY{p}{(}\PY{l+s+s2}{\PYZdq{}}\PY{l+s+s2}{mllib\PYZus{}classifier}\PY{l+s+s2}{\PYZdq{}}\PY{p}{)} \PYZbs{}
                 \PY{o}{.}\PY{n}{getOrCreate}\PY{p}{(}\PY{p}{)}
\end{Verbatim}


    \begin{Verbatim}[commandchars=\\\{\}]
{\color{incolor}In [{\color{incolor}15}]:} \PY{n}{spark}
\end{Verbatim}


\begin{Verbatim}[commandchars=\\\{\}]
{\color{outcolor}Out[{\color{outcolor}15}]:} <pyspark.sql.session.SparkSession at 0x7fe68eae7668>
\end{Verbatim}
            
    \begin{Verbatim}[commandchars=\\\{\}]
{\color{incolor}In [{\color{incolor}16}]:} \PY{n}{sc} \PY{o}{=} \PY{n}{spark}\PY{o}{.}\PY{n}{sparkContext}
\end{Verbatim}


    \begin{Verbatim}[commandchars=\\\{\}]
{\color{incolor}In [{\color{incolor}18}]:} \PY{n}{spam} \PY{o}{=} \PY{n}{sc}\PY{o}{.}\PY{n}{textFile}\PY{p}{(}\PY{n}{os}\PY{o}{.}\PY{n}{path}\PY{o}{.}\PY{n}{join}\PY{p}{(}\PY{n}{data\PYZus{}path}\PY{p}{,} \PY{l+s+s2}{\PYZdq{}}\PY{l+s+s2}{spam.txt}\PY{l+s+s2}{\PYZdq{}}\PY{p}{)}\PY{p}{)}
         \PY{n}{ham} \PY{o}{=} \PY{n}{sc}\PY{o}{.}\PY{n}{textFile}\PY{p}{(}\PY{n}{os}\PY{o}{.}\PY{n}{path}\PY{o}{.}\PY{n}{join}\PY{p}{(}\PY{n}{data\PYZus{}path}\PY{p}{,} \PY{l+s+s2}{\PYZdq{}}\PY{l+s+s2}{ham.txt}\PY{l+s+s2}{\PYZdq{}}\PY{p}{)}\PY{p}{)}
\end{Verbatim}


    \begin{Verbatim}[commandchars=\\\{\}]
{\color{incolor}In [{\color{incolor} }]:} \PY{n}{spam}\PY{o}{.}\PY{n}{collect}\PY{p}{(}\PY{p}{)}
\end{Verbatim}


    \begin{Verbatim}[commandchars=\\\{\}]
{\color{incolor}In [{\color{incolor} }]:} \PY{n}{ham}\PY{o}{.}\PY{n}{collect}\PY{p}{(}\PY{p}{)}
\end{Verbatim}


    \begin{Verbatim}[commandchars=\\\{\}]
{\color{incolor}In [{\color{incolor}22}]:} \PY{n}{tf} \PY{o}{=} \PY{n}{HashingTF}\PY{p}{(}\PY{n}{numFeatures} \PY{o}{=} \PY{l+m+mi}{10000}\PY{p}{)}
\end{Verbatim}


    \begin{Verbatim}[commandchars=\\\{\}]
{\color{incolor}In [{\color{incolor}23}]:} \PY{n}{spamFeatures} \PY{o}{=} \PY{n}{spam}\PY{o}{.}\PY{n}{map}\PY{p}{(}\PY{k}{lambda} \PY{n}{email}\PY{p}{:} \PY{n}{tf}\PY{o}{.}\PY{n}{transform}\PY{p}{(}\PY{n}{email}\PY{o}{.}\PY{n}{split}\PY{p}{(}\PY{l+s+s2}{\PYZdq{}}\PY{l+s+s2}{ }\PY{l+s+s2}{\PYZdq{}}\PY{p}{)}\PY{p}{)}\PY{p}{)}
         \PY{n}{normalFeatures} \PY{o}{=} \PY{n}{ham}\PY{o}{.}\PY{n}{map}\PY{p}{(}\PY{k}{lambda} \PY{n}{email}\PY{p}{:} \PY{n}{tf}\PY{o}{.}\PY{n}{transform}\PY{p}{(}\PY{n}{email}\PY{o}{.}\PY{n}{split}\PY{p}{(}\PY{l+s+s2}{\PYZdq{}}\PY{l+s+s2}{ }\PY{l+s+s2}{\PYZdq{}}\PY{p}{)}\PY{p}{)}\PY{p}{)}
\end{Verbatim}


    \begin{Verbatim}[commandchars=\\\{\}]
{\color{incolor}In [{\color{incolor}24}]:} \PY{c+c1}{\PYZsh{} Build LabeledPoint datasets (1=spam, 0=ham)}
         \PY{n}{positiveExamples} \PY{o}{=} \PY{n}{spamFeatures}\PY{o}{.}\PY{n}{map}\PY{p}{(}\PY{k}{lambda} \PY{n}{features}\PY{p}{:} \PY{n}{LabeledPoint}\PY{p}{(}\PY{l+m+mi}{1}\PY{p}{,} \PY{n}{features}\PY{p}{)}\PY{p}{)}
         \PY{n}{negativeExamples} \PY{o}{=} \PY{n}{normalFeatures}\PY{o}{.}\PY{n}{map}\PY{p}{(}\PY{k}{lambda} \PY{n}{features}\PY{p}{:} \PY{n}{LabeledPoint}\PY{p}{(}\PY{l+m+mi}{0}\PY{p}{,} \PY{n}{features}\PY{p}{)}\PY{p}{)}
\end{Verbatim}


    \begin{Verbatim}[commandchars=\\\{\}]
{\color{incolor}In [{\color{incolor}25}]:} \PY{n}{pos} \PY{o}{=} \PY{n}{positiveExamples}\PY{o}{.}\PY{n}{collect}\PY{p}{(}\PY{p}{)}
\end{Verbatim}


    \begin{Verbatim}[commandchars=\\\{\}]
{\color{incolor}In [{\color{incolor} }]:} \PY{n}{pos}\PY{p}{[}\PY{l+m+mi}{0}\PY{p}{]}
\end{Verbatim}


    \begin{Verbatim}[commandchars=\\\{\}]
{\color{incolor}In [{\color{incolor}28}]:} \PY{n}{neg} \PY{o}{=} \PY{n}{negativeExamples}\PY{o}{.}\PY{n}{collect}\PY{p}{(}\PY{p}{)}
\end{Verbatim}


    \begin{Verbatim}[commandchars=\\\{\}]
{\color{incolor}In [{\color{incolor} }]:} \PY{n}{neg}\PY{p}{[}\PY{l+m+mi}{0}\PY{p}{]}
\end{Verbatim}


    \begin{Verbatim}[commandchars=\\\{\}]
{\color{incolor}In [{\color{incolor} }]:} \PY{c+c1}{\PYZsh{} Build training set}
        \PY{n}{trainData} \PY{o}{=} \PY{n}{positiveExamples}\PY{o}{.}\PY{n}{union}\PY{p}{(}\PY{n}{negativeExamples}\PY{p}{)}
        \PY{n}{trainData}\PY{o}{.}\PY{n}{cache}\PY{p}{(}\PY{p}{)}
\end{Verbatim}


    \begin{Verbatim}[commandchars=\\\{\}]
{\color{incolor}In [{\color{incolor}31}]:} \PY{c+c1}{\PYZsh{} Train LogReg model}
         \PY{n}{model} \PY{o}{=} \PY{n}{LogisticRegressionWithSGD}\PY{o}{.}\PY{n}{train}\PY{p}{(}\PY{n}{trainData}\PY{p}{)}
\end{Verbatim}


    \begin{Verbatim}[commandchars=\\\{\}]
{\color{incolor}In [{\color{incolor} }]:} \PY{c+c1}{\PYZsh{} push \PYZdq{}not spam\PYZdq{} example through classifier}
        \PY{n}{posTest} \PY{o}{=} \PY{n}{tf}\PY{o}{.}\PY{n}{transform}\PY{p}{(}\PY{l+s+s2}{\PYZdq{}}\PY{l+s+s2}{I love learning Spark programming}\PY{l+s+s2}{\PYZdq{}}\PY{o}{.}\PY{n}{split}\PY{p}{(}\PY{l+s+s2}{\PYZdq{}}\PY{l+s+s2}{ }\PY{l+s+s2}{\PYZdq{}}\PY{p}{)}\PY{p}{)}
        \PY{n}{posTest}
\end{Verbatim}


    \begin{Verbatim}[commandchars=\\\{\}]
{\color{incolor}In [{\color{incolor}38}]:} \PY{c+c1}{\PYZsh{} Prediction}
         \PY{n+nb}{print}\PY{p}{(}\PY{l+s+s2}{\PYZdq{}}\PY{l+s+s2}{Prediction for positive example: }\PY{l+s+si}{\PYZob{}\PYZcb{}}\PY{l+s+s2}{\PYZdq{}}\PY{o}{.}\PY{n}{format}\PY{p}{(}\PY{n}{model}\PY{o}{.}\PY{n}{predict}\PY{p}{(}\PY{n}{posTest}\PY{p}{)}\PY{p}{)}\PY{p}{)}
\end{Verbatim}


    \begin{Verbatim}[commandchars=\\\{\}]
Prediction for positive example: 0

    \end{Verbatim}

    \textbf{LabeledPoint}\\
Stores feature vector together with label\\
\textbf{Rating}\\
Rating of product by a user. Used in recommendation, for instance.\\
\textbf{Vector}\\
Handles dense and sparse. For sparse, only nonzero values and their
indices are stored.\\
Sparse saves on memory and runtime.\\
\textbf{Matrix}\\
A local matrix has integer-typed row and column indices and double-typed
values, stored on a single machine.\\
MLlib supports dense matrices, whose entry values are stored in a single
double array in column-major order, and sparse matrices, whose non-zero
entry values are stored in the Compressed Sparse Column (CSC) format in
column-major order.\\
\textbf{Distributed matrix}\\
A distributed matrix has long-typed row and column indices and
double-typed values\\
\textbf{Row matrix}\\
A RowMatrix is a row-oriented distributed matrix without meaningful row
indices\\
\textbf{CoordinateMatrix}\\
CoordinateMatrix is a distributed matrix backed by an RDD of its
entries\\
A CoordinateMatrix should be used only when both dimensions of the
matrix are huge and the matrix is very sparse.

https://en.wikipedia.org/wiki/Row-\_and\_column-major\_order

    \begin{Verbatim}[commandchars=\\\{\}]
{\color{incolor}In [{\color{incolor}41}]:} \PY{c+c1}{\PYZsh{} Build sparse vector}
         \PY{k+kn}{from} \PY{n+nn}{pyspark}\PY{n+nn}{.}\PY{n+nn}{mllib}\PY{n+nn}{.}\PY{n+nn}{linalg} \PY{k}{import} \PY{n}{Vectors}
         
         \PY{c+c1}{\PYZsh{} create sparse vector [1.0 0.0 2.0 0.0]}
         \PY{n}{sv1} \PY{o}{=} \PY{n}{Vectors}\PY{o}{.}\PY{n}{sparse}\PY{p}{(}\PY{l+m+mi}{4}\PY{p}{,} \PY{p}{\PYZob{}}\PY{l+m+mi}{0}\PY{p}{:} \PY{l+m+mf}{1.0}\PY{p}{,} \PY{l+m+mi}{2}\PY{p}{:} \PY{l+m+mf}{2.0}\PY{p}{\PYZcb{}}\PY{p}{)}
\end{Verbatim}


    \begin{Verbatim}[commandchars=\\\{\}]
{\color{incolor}In [{\color{incolor} }]:} \PY{n}{sv1}
\end{Verbatim}


    \hypertarget{feature-extraction}{%
\subsubsection{Feature Extraction}\label{feature-extraction}}

\emph{mllib.feature}\\
contains classes for common feature transformations:\\
- Term Frequency-Inverse Document Frequency (TF-IDF)\\
Produces feature vectors from text documents

There are two algorithms that compute TF-IDF:

\textbf{1. HashingTF}\\
Computes term frequency vector from document\\
Can process one document or an RDD of documents\\
Each document needs to be an interable sequence (a list in Python)

To reduce the chance of collision, we can increase the target feature
dimension, i.e., the\\
number of buckets of the hash table. The default feature dimension is
1,048,576

\textbf{2. IDF}\\
Computes inverse document frequency\\
Terms that appear in high fraction of the docs are not as valuable\\
IDF will downweight such terms

    Good example of Feature Extraction here:\\
http://spark.apache.org/docs/1.2.0/mllib-feature-extraction.html

    \textbf{TF-IDF Example}

    \textbf{Word2Vec}\\
Computes distributed vector representation of words.\\
Similar words are close in the vector space\\
Useful in many NLP applications:\\
named entity recognition, disambiguation, parsing, tagging and machine
translation.

    \hypertarget{fit-word2vecmodel-to-some-text-data}{%
\subsubsection{Fit Word2VecModel to some text
data}\label{fit-word2vecmodel-to-some-text-data}}

    \begin{Verbatim}[commandchars=\\\{\}]
{\color{incolor}In [{\color{incolor} }]:} \PY{k+kn}{from} \PY{n+nn}{pyspark}\PY{n+nn}{.}\PY{n+nn}{mllib}\PY{n+nn}{.}\PY{n+nn}{feature} \PY{k}{import} \PY{n}{Word2Vec}
        
        \PY{n}{inp} \PY{o}{=} \PY{n}{sc}\PY{o}{.}\PY{n}{textFile}\PY{p}{(}\PY{l+s+s2}{\PYZdq{}}\PY{l+s+s2}{C:/spark/spark\PYZhy{}2.1.1\PYZhy{}bin\PYZhy{}hadoop2.7/data/text8\PYZus{}part1.txt}\PY{l+s+s2}{\PYZdq{}}\PY{p}{)}\PY{o}{.}\PY{n}{map}\PY{p}{(}\PY{k}{lambda} \PY{n}{row}\PY{p}{:} \PY{n}{row}\PY{o}{.}\PY{n}{split}\PY{p}{(}\PY{l+s+s2}{\PYZdq{}}\PY{l+s+s2}{ }\PY{l+s+s2}{\PYZdq{}}\PY{p}{)}\PY{p}{)}
        \PY{n}{word2vec} \PY{o}{=} \PY{n}{Word2Vec}\PY{p}{(}\PY{p}{)}
        \PY{n}{model} \PY{o}{=} \PY{n}{word2vec}\PY{o}{.}\PY{n}{fit}\PY{p}{(}\PY{n}{inp}\PY{p}{)}
        
        \PY{n}{synonyms} \PY{o}{=} \PY{n}{model}\PY{o}{.}\PY{n}{findSynonyms}\PY{p}{(}\PY{l+s+s1}{\PYZsq{}}\PY{l+s+s1}{china}\PY{l+s+s1}{\PYZsq{}}\PY{p}{,} \PY{l+m+mi}{40}\PY{p}{)}
        
        \PY{k}{for} \PY{n}{word}\PY{p}{,} \PY{n}{cosine\PYZus{}distance} \PY{o+ow}{in} \PY{n}{synonyms}\PY{p}{:}
            \PY{n+nb}{print}\PY{p}{(}\PY{l+s+s2}{\PYZdq{}}\PY{l+s+si}{\PYZob{}\PYZcb{}}\PY{l+s+s2}{: }\PY{l+s+si}{\PYZob{}\PYZcb{}}\PY{l+s+s2}{\PYZdq{}}\PY{o}{.}\PY{n}{format}\PY{p}{(}\PY{n}{word}\PY{p}{,} \PY{n}{cosine\PYZus{}distance}\PY{p}{)}\PY{p}{)}
        \PY{n}{Top} \PY{n}{Records}\PY{p}{:}
        \PY{n}{malaysia}\PY{p}{:} \PY{l+m+mf}{0.9055396899188917}
        \PY{n}{eastern}\PY{p}{:} \PY{l+m+mf}{0.8834685956632131}
        \PY{n}{africa}\PY{p}{:} \PY{l+m+mf}{0.8537198739068056}
        \PY{n}{zambia}\PY{p}{:} \PY{l+m+mf}{0.8535407384161012}
        \PY{n}{myanmar}\PY{p}{:} \PY{l+m+mf}{0.8475548784366893}
        \PY{n}{predominantly}\PY{p}{:} \PY{l+m+mf}{0.8461926971224027}
        \PY{n}{mongolia}\PY{p}{:} \PY{l+m+mf}{0.8371518611342739}
        \PY{n}{countries}\PY{p}{:} \PY{l+m+mf}{0.8342705781501009}
        \PY{n}{southeast}\PY{p}{:} \PY{l+m+mf}{0.8316274754770454}
        \PY{n}{central}\PY{p}{:} \PY{l+m+mf}{0.8313670331856243}
\end{Verbatim}


    \textbf{StandardScaler}

Standardization can improve the convergence rate during the optimization
process, and also\\
prevents against features with very large variances exerting an overly
large influence during model training.

For each feature,\\
1. Scales to unit variance\\
2. Centers to mean zero\\
Useful or even essential for some models\\
K-means works in Euclidean space, so all features should be on same
scale\\
Tree models do not need this

Use this in a \emph{Pipeline} so the statistics can be applied to
datasets for scoring later

    \hypertarget{standard-scaler}{%
\subsubsection{Standard Scaler}\label{standard-scaler}}

Load dataset in libsvm format, standardize the features so that the new
features have unit variance and/or zero mean

    \begin{Verbatim}[commandchars=\\\{\}]
{\color{incolor}In [{\color{incolor}43}]:} \PY{k+kn}{from} \PY{n+nn}{pyspark}\PY{n+nn}{.}\PY{n+nn}{mllib}\PY{n+nn}{.}\PY{n+nn}{util} \PY{k}{import} \PY{n}{MLUtils}
         \PY{k+kn}{from} \PY{n+nn}{pyspark}\PY{n+nn}{.}\PY{n+nn}{mllib}\PY{n+nn}{.}\PY{n+nn}{linalg} \PY{k}{import} \PY{n}{Vectors}
         \PY{k+kn}{from} \PY{n+nn}{pyspark}\PY{n+nn}{.}\PY{n+nn}{mllib}\PY{n+nn}{.}\PY{n+nn}{feature} \PY{k}{import} \PY{n}{StandardScaler}
\end{Verbatim}


    \begin{Verbatim}[commandchars=\\\{\}]
{\color{incolor}In [{\color{incolor}47}]:} \PY{n}{data} \PY{o}{=} \PY{n}{MLUtils}\PY{o}{.}\PY{n}{loadLibSVMFile}\PY{p}{(}\PY{n}{sc}\PY{p}{,} \PY{n}{os}\PY{o}{.}\PY{n}{path}\PY{o}{.}\PY{n}{join}\PY{p}{(}\PY{n}{data\PYZus{}path}\PY{p}{,} \PY{l+s+s2}{\PYZdq{}}\PY{l+s+s2}{sample\PYZus{}libsvm\PYZus{}data.txt}\PY{l+s+s2}{\PYZdq{}}\PY{p}{)}\PY{p}{)}
\end{Verbatim}


    \begin{Verbatim}[commandchars=\\\{\}]
{\color{incolor}In [{\color{incolor} }]:} \PY{n}{data}\PY{o}{.}\PY{n}{take}\PY{p}{(}\PY{l+m+mi}{1}\PY{p}{)}
\end{Verbatim}


    \begin{Verbatim}[commandchars=\\\{\}]
{\color{incolor}In [{\color{incolor} }]:} \PY{n+nb}{type}\PY{p}{(}\PY{n}{data}\PY{p}{)}
\end{Verbatim}


    \begin{Verbatim}[commandchars=\\\{\}]
{\color{incolor}In [{\color{incolor}51}]:} \PY{c+c1}{\PYZsh{} extract labels and features; stored as RDDs}
         \PY{n}{label} \PY{o}{=} \PY{n}{data}\PY{o}{.}\PY{n}{map}\PY{p}{(}\PY{k}{lambda} \PY{n}{x}\PY{p}{:} \PY{n}{x}\PY{o}{.}\PY{n}{label}\PY{p}{)}
         \PY{n}{features} \PY{o}{=} \PY{n}{data}\PY{o}{.}\PY{n}{map}\PY{p}{(}\PY{k}{lambda} \PY{n}{x}\PY{p}{:} \PY{n}{x}\PY{o}{.}\PY{n}{features}\PY{p}{)}
\end{Verbatim}


    \begin{Verbatim}[commandchars=\\\{\}]
{\color{incolor}In [{\color{incolor}53}]:} \PY{n}{scaler1} \PY{o}{=} \PY{n}{StandardScaler}\PY{p}{(}\PY{p}{)}\PY{o}{.}\PY{n}{fit}\PY{p}{(}\PY{n}{features}\PY{p}{)}
\end{Verbatim}


    \begin{Verbatim}[commandchars=\\\{\}]
{\color{incolor}In [{\color{incolor}54}]:} \PY{c+c1}{\PYZsh{} data1 will be unit variance.}
         \PY{n}{data1} \PY{o}{=} \PY{n}{label}\PY{o}{.}\PY{n}{zip}\PY{p}{(}\PY{n}{scaler1}\PY{o}{.}\PY{n}{transform}\PY{p}{(}\PY{n}{features}\PY{p}{)}\PY{p}{)}
\end{Verbatim}


    \begin{Verbatim}[commandchars=\\\{\}]
{\color{incolor}In [{\color{incolor} }]:} \PY{n}{data1}\PY{o}{.}\PY{n}{take}\PY{p}{(}\PY{l+m+mi}{2}\PY{p}{)}
\end{Verbatim}



    % Add a bibliography block to the postdoc
    
    
    
    \end{document}
